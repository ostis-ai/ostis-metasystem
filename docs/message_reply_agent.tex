\scnheader{Агент ответа на сообщение}

\scntext{примечание}{Данный агент генерирует ответное сообщение, связанное с исходным сообщением с помощью отношения nrel\_reply}

\scntext{класс действия}{action\_reply\_to\_message}

\begin{scnrelfromvector}{задачи}
  \scnfileitem{сформировать ответ на сообщение}
\end{scnrelfromvector}

\begin{scnrelfromvector}{аргументы агента}
  \scnitem{linkAddr}
  \begin{scnindent}
  \scntext{пояснение}{linkAddr - sc-ссылка с текстом сообщения пользователя}
  \end{scnindent}
  \scnitem{processingProgramAddr}
  \begin{scnindent}
  \scntext{пояснение}{processingProgramAddr - sc-узел обрабатывающей программы}
  \end{scnindent}
\end{scnrelfromvector}

\begin{scnrelfromvector}{алгоритм}
  \scnfileitem{Генерирация узла сообщения в базе знаний, идентификация полученного текстового файла как текст этого сообщения}
  \scnfileitem{Генерирация необходимой конструкции для вызова агента интерпретации неатомарного действия. Пример этой конструкции показан ниже}
  \scnfileimage[20em]{images/message-reply-agent/message_processing_program_example.png}
  \scnfileitem{Ожидание завершения работы агента интерпретации и выполнение поиска ответного сообщения, сгенерированное во время работы агента интерпретации}
  \scnfileitem{Добавление ответного сообщения к ответной структуре}
\end{scnrelfromvector}

\scnrelfrom{пример входной конструкции}{\scnfileimage[30em]{images/message-reply-agent/message_reply_input.png}}

\scnrelfrom{пример выходной конструкции}{\scnfileimage[30em]{images/message-reply-agent/message_reply_output.png}}

\scntext{язык реализации агента}{c++}

\begin{scnrelfromvector}{возможные результаты}
  \scnitem{SC\_RESULT\_OK}
  \begin{scnindent}
  \scntext{пояснение}{ответное сообщение сгенерировано}
  \end{scnindent}
  \scnitem{SC\_RESULT\_ERROR}
  \begin{scnindent}
  \scntext{пояснение}{внутренняя ошибка}
  \end{scnindent}
\end{scnrelfromvector}

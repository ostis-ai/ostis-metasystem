\scnheader{Агент перевода основных и системных идентификаторов узлов из sc-памяти в текстовый файл}

\scnidtf{sc-агент трансляции идентификаторов узлов из sc-памяти в текстовый файл}
\begin{scnrelfromvector}{задачи}
   \scnitem{поиск системных и основных идентификаторов узлов в sc-памяти}
   \scnitem{проверка узлов на наличие только одного системного идентификатора и одного основного идентификатора на русском языке}
   \scnitem{трансляция в текстовый файл является}
\end{scnrelfromvector}
\begin{scnrelfromvector}{аргументы агента}
	\scnitem{пустое множество}
\end{scnrelfromvector}
\scntext{алгоритм}
{
\begin{scnitemize}
    \item Поиск всех узлов с помощью итератора, который ищет все конструкции вида \scnfileimage[20em]{images/translate_agent_alg_1.png}
    \item Проверка каждого узла на выполнение трех условий:
        \begin{scnitemizeii}
            \item Наличие только одного системного идентификатора.
            \item Наличие только одного основного русского идентификатора.
            \item Принадлежность одному из sc-типов узлов.
        \end{scnitemizeii}
    \item Если не выполняется одно из условий, то запись данных об узле в файл не выполняется.
    \item Если у узла более одного системного идентификатора, то вызывается исключение.
    \item Если все три условия выполняются, то данные об узле записываются в файл.
    \item Если произошла ошибка при работе с файлом, вызывается исключение.
\end{scnitemize}
}
\scntext{ответ агента}
{В результате агент создает текстовый файл, в котором в виде словаря формируются структуры.
Роль ключа играет основной русский идентификатор, роль значения -- пара, в которой на первом месте стоит системный идентификатор, а на втором -- sc-тип узла.}
\begin{scnindent}
	\scntext{пример}
	{\{\scnqqi{main\_ru\_identifier}, \{\scnqqi{system\_identifier}, \scnqqi{sc\_type}\} \}}
\end{scnindent}
\scnrelfrom{пример входной конструкции}{\scnfileimage[30em]{images/translate_agent_input.png}}
\scnrelfrom{пример выходной конструкции}{\scnfileimage[30em]{images/translate_agent_output.png}}

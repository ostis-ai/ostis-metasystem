\scnheader{Агент классификации класса сообщения}

\scnidtf{sc-агент классификации класса сообщения в sc-памяти}
\begin{scnrelfromvector}{задачи}
  \scnitem{определить класс сообщения}
\end{scnrelfromvector}

\begin{scnrelfromvector}{аргументы агента}
  \scnitem{сообщение}
\end{scnrelfromvector}

\begin{scnrelfromvector}{алгоритм}
  \scnfileitem{Получение текста сообщения}
  \scnfileitem{Получение темы сообщения}
  \scnfileitem{Получение признаков сообщения}
  \scnfileitem{Получение сущности сообщения}
  \scnfileitem{Если что-то не получилось получить, то происходит завершение работы агента}
  \scnfileitem{Если агент завершил работу успешно, то его ответ прикрепляется к сообщению отношением \textit{nrel\_entity}}
  \scnfileitem {Если агент завершил работу неуспешно, то происходит завершение работы текущего агента}
\end{scnrelfromvector}

\scnrelfrom{пример входной конструкции}{\scnfileimage[30em]{images/message-topic-classification-agent/input.png}}

\scnrelfrom{пример выходной конструкции}{\scnfileimage[30em]{images/message-topic-classification-agent/output.png}}

\scnrelfrom{Пример структуры, необходимой для классификации сообщения по теме}{\scnfileimage[30em]{images/message-topic-classification-agent/intent.png}}

\scnrelfrom{Пример структуры, необходимой для классификации сообщения по признакам}{\scnfileimage[30em]{images/message-topic-classification-agent/trait.png}}

\scnrelfrom{Пример структуры, необходимой для получения сущностей сообщения}{\scnfileimage[30em]{images/message-topic-classification-agent/entity.png}}